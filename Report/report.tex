% uw-wkrpt-se.tex - An example work report that uses uw-wkrpt.cls
% Copyright (C) 2002,2003  Simon Law
% 
% This program is free software; you can redistribute it and/or modify
% it under the terms of the GNU General Public License as published by
% the Free Software Foundation; either version 2 of the License, or
% (at your option) any later version.
% 
% This program is distributed in the hope that it will be useful,
% but WITHOUT ANY WARRANTY; without even the implied warranty of
% MERCHANTABILITY or FITNESS FOR A PARTICULAR PURPOSE.  See the
% GNU General Public License for more details.
% 
% You should have received a copy of the GNU General Public License
% along with this program; if not, write to the Free Software
% Foundation, Inc., 59 Temple Place, Suite 330, Boston, MA  02111-1307  USA
%
%%%%%%%%%%%%%%%%%%%%%%%%%%%%%%%%%%%%%%%%%%%%%%%%%%%%%%%%%%%%%%%%%%%%%
%
% We begin by calling the workreport class which includes all the
% definitions for the macros we will use.
\documentclass[se]{uw-wkrpt}

% LaTeX preamble: load some packages to add functionality
\usepackage{graphicx} % Include graphic importing

\usepackage[T1]{fontenc} % Better fonts
\usepackage{ae,aecompl}

\usepackage{indentfirst} % Indent first paragraph of each section

\usepackage[titletoc,title]{appendix} % Prefix appendix letters with `Appendix'

% For mathematical symbols in our pseudocode
\usepackage{amsmath}

% Use the algorithmicx package for pseudocode
\usepackage{algorithm}
\usepackage{algpseudocode}

% Use biblatex for references
\usepackage[style=ieee,sorting=none,dateabbrev=false,backend=biber]{biblatex}
\addbibresource{uw-wkrpt-bib.bib} % Specify the bibliography file

% This needs to be the last package loaded
\usepackage[pdftex]{hyperref} % Generate PDF links and bookmarks.
\hypersetup{
  bookmarks=true,
  bookmarksnumbered=true
}

% Now we will begin writing the document.
\begin{document}

%%%%%%%%%%%%%%%%%%%%%%%%%%%%%%%%%%%%%%%%%%%%%%%%%%%%%%%%%%%%%%%%%%%%%
%% IMPORTANT INFORMATION
%%%%%%%%%%%%%%%%%%%%%%%%%%%%%%%%%%%%%%%%%%%%%%%%%%%%%%%%%%%%%%%%%%%%%

%% First we, should create a title page.  This is done below:
% Fill in the title of your report.
\title{OS WOW RTX Documentation}

% Fill in your name.
\author{Fasih Awan, Joshua Kalpin, Si Chuang Li, John Zanutto}

% Fill in your student ID number.
\uwid{TODO, 20414492, TODO, TODO}

% Fill in the name of the PNG file with your signature, or leave unchanged.
\signature{signature}

% Fill in your home address.
\address{123 University Ave. W.\\*
         Waterloo, ON\ \ N2L 3G1}

% Fill in your employer's name.
\employer{SE 350}

% Fill in your employer's city and province.
\employeraddress{Thomas Reidemeister}

% Fill in your school's name.
\school{University of Waterloo}

% Fill in your faculty name.
\faculty{Software Engineering}

% Fill in your student user ID
\userid{fawan, jkalpin, TODO, jzanutto}

% Fill in your e-mail address.
\email{jrhacker@engmail}

% Fill in your term.
\term{3A}

% Fill in your program.
\program{Software Engineering}

% Fill in the department chair's name.
\chair{Dr.\ A.\ Morton}

% Fill in the department chair's mailing address.
\chairaddress{Software Engineering\\*
              University of Waterloo\\*
	      Waterloo, ON\ \ N2L 3G1}

% If you are writing an "SE-confidential" report, uncomment the next line.
%\confidential{SE-confidential}

% If you want to specify the date, fill it in here.  If you comment out
% this line, today's date will be substituted.
%\date{April 24, 2012}

% Now, we ask LaTeX to generate the title.
\maketitle

%%%%%%%%%%%%%%%%%%%%%%%%%%%%%%%%%%%%%%%%%%%%%%%%%%%%%%%%%%%%%%%%%%%%%
%% FRONT MATTER
%%%%%%%%%%%%%%%%%%%%%%%%%%%%%%%%%%%%%%%%%%%%%%%%%%%%%%%%%%%%%%%%%%%%%
%% \frontmatter will make the \section commands ignore their numbering,
%% it will also use roman page numbers.
\frontmatter

% We continue with required sections, such as the Executive Summary.
\section{Executive Summary}
TODO

% Next, we need to make a Table of Contents, List of Figures and 
% List of Tables.  You will most likely need to run LaTeX twice to
% get these correct.  The first pass for LaTeX to figure out the
% labels, and the second pass to put in the right references.
\tableofcontents
\listoffigures
\listoftables

%%%%%%%%%%%%%%%%%%%%%%%%%%%%%%%%%%%%%%%%%%%%%%%%%%%%%%%%%%%%%%%%%%%%%
%% REPORT BODY
%%%%%%%%%%%%%%%%%%%%%%%%%%%%%%%%%%%%%%%%%%%%%%%%%%%%%%%%%%%%%%%%%%%%%
%% \main will make the \section commands numbered again,
%% it will also use arabic page numbers.
\mainmatter

% You must have an Introduction
\section{Introduction}\label{sec:intro}

\section{Global Variables}\label{sec:global}

\section{Kernel API}\label{sec:kernel}

\section{Interrupts}\label{sec:interupt}

\section{System and User Processes}\label{sec:proc}

\section{Test Processes}\label{sec:testproc}

\section{Initialization}\label{sec:init}

\section{Testing}\label{sec:test}

\section{Major Design Challenges}\label{sec:design}

\section{Timing Analysis}\label{sec:time}

\end{document}
