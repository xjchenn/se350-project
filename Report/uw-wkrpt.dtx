% \iffalse meta-comment
% uw-wkrpt.dtx - DocStrip source code for the uw-wkrpt package
% Copyright (C) 2003  Simon Law
% 
%   This program is free software; you can redistribute it and/or modify
%   it under the terms of the GNU General Public License as published by
%   the Free Software Foundation; either version 2 of the License, or
%   (at your option) any later version.
% 
%   This program is distributed in the hope that it will be useful,
%   but WITHOUT ANY WARRANTY; without even the implied warranty of
%   MERCHANTABILITY or FITNESS FOR A PARTICULAR PURPOSE.  See the
%   GNU General Public License for more details.
% 
%   You should have received a copy of the GNU General Public License
%   along with this program; if not, write to the Free Software
%   Foundation, Inc., 59 Temple Place, Suite 330, Boston, MA  02111-1307  USA
%
% \fi
% \def\fileversion{v2.8.3}
% \def\filedate{2012/09/16}
% \iffalse
%<class>\NeedsTeXFormat{LaTeX2e}
%<class>\ProvidesPackage{uw-wkrpt}[2012/09/09 v2.8.2 UWaterloo work reports]
%<*driver>
\NeedsTeXFormat{LaTeX2e}[1995/12/01]
\documentclass[letterpaper]{ltxdoc}
\usepackage{calc}
\usepackage{url} \urlstyle{sf}
\usepackage{textcomp}
\usepackage{fancyvrb}
\usepackage{multicol}
\newlength{\ExampleWidth}
\setlength{\ExampleWidth}{3.5cm}
\fvset{gobble=6,numbersep=3pt,frame=single,
       numbers=left,xleftmargin=5mm,xrightmargin=0pt}
\fvset{xrightmargin=\ExampleWidth}
\EnableCrossrefs
 %\DisableCrossrefs     % Say \DisableCrossrefs if index is ready
\CodelineIndex
\RecordChanges          % Gather update information
 %\OnlyDescription      % comment out for implementation details
 \OldMakeindex           % use if your MakeIndex is pre-v2.9
 %\setlength\hfuzz{15pt}  % don't make so many
 %\hbadness=7000          % over and under full box warnings
\begin{document}
  \DocInput{uw-wkrpt.dtx}
\end{document}
%</driver>
% \fi
%
% \StopEventually{\PrintIndex \PrintChanges}
% \CheckSum{1135}
%
% \DoNotIndex{\#,\$,\#,\&,\@,\\,\{,\},\^,\_,\~,\ }
% \DoNotIndex{\addcontentsline, \addpenalty, \addtolength, \advance, \and}
% \DoNotIndex{\begin, \begingroup, \bfseries, \bibliographystyle, \boolean}
% \DoNotIndex{\ClassError, \ClassWarning, \clearpage, \columnwidth, 
%             \contentsname, \csname, \c@secnumdepth, \c@tocdepth, 
%             \CurrentOption}
% \DoNotIndex{\DeclareRobustCommand, \DeclareOption, \def, \@dotsep}
% \DoNotIndex{\@empty, \else, \end, \endcsname, \endgroup, \equal,
%             \evensidemargin, \everyvbox}
% \DoNotIndex{\fi, \footnoterule, \footnotesize, \footskip}
% \DoNotIndex{\gdef, \global, \@gobble}
% \DoNotIndex{\@hangfrom, \hb@xt@, \hbox, \headheight, \headsep, 
%             \@highpenalty, \hfil, \hskip, \hss, \huge}
% \DoNotIndex{\ifdim, \ifnum, \ifthenelse, \ifx, \interlinepenalty}
% \DoNotIndex{\large, \leaders, \leavemode, \leavevmode, \lengthtest, 
%             \leftskip, \let, \LoadClass, \lowercase, \l@section}
% \DoNotIndex{\@M, \m@ne, \m@th, \makebox, \MakeTextUppercase, 
%             \medskipamount, \mkern}
% \DoNotIndex{\newcommand, \newenvironment, \newif, \newlength, \nobreak, 
%             \noexpand, \noindent, \null, \numberline}
% \DoNotIndex{\oddsidemargin, \or}
% \DoNotIndex{\p@, \paperheight, \paperwidth, \@@par, \par, \parbox, 
%             \parfillskip, \parindent, \PassOptionsToClass, \penalty, 
%             \@plus, \@pnumwidth, \ProcessOptions, \protect, 
%             \protected@edef, \providecommand}
% \DoNotIndex{\relax, \refstepcounter, \renewcommand, \renewenvironment, 
%             \RequirePackage, \rightskip, \rule}
% \DoNotIndex{\@svsec, \@svsechd, \@seccntformat, \@sect, \setcounter, 
%             \setlength, \settowidth, \small, \string}
% \DoNotIndex{\@tempdima, \@tempskipa, \textbf, \textheight, \textit, 
%             \textsc, \textwidth, \thanks, \thesection, \thispagestyle,
%             \today, \topmargin}
% \DoNotIndex{\undefined, \urlstyle, \usepackage, 
%             \UWECEWorkReportVersion, \uwwkrpt@ecefalse, 
%             \uwwkrpt@ecetrue, \uwwkrpt@mathfalse, \uwwkrpt@mathtrue, 
%             \uwwkrpt@sefalse, \uwwkrpt@setrue}
% \DoNotIndex{\vfill, \vskip, \vspace}
% \DoNotIndex{\write}
% \DoNotIndex{\@xsect}
% \DoNotIndex{\z@}
% \setcounter{IndexColumns}{2}
%
% \CharacterTable
%  {Upper-case    \A\B\C\D\E\F\G\H\I\J\K\L\M\N\O\P\Q\R\S\T\U\V\W\X\Y\Z
%   Lower-case    \a\b\c\d\e\f\g\h\i\j\k\l\m\n\o\p\q\r\s\t\u\v\w\x\y\z
%   Digits        \0\1\2\3\4\5\6\7\8\9
%   Exclamation   \!     Double quote  \"     Hash (number) \#
%   Dollar        \$     Percent       \%     Ampersand     \&
%   Acute accent  \'     Left paren    \(     Right paren   \)
%   Asterisk      \*     Plus          \+     Comma         \,
%   Minus         \-     Point         \.     Solidus       \/
%   Colon         \:     Semicolon     \;     Less than     \<
%   Equals        \=     Greater than  \>     Question mark \?
%   Commercial at \@     Left bracket  \[     Backslash     \\
%   Right bracket \]     Circumflex    \^     Underscore    \_
%   Grave accent  \`     Left brace    \{     Vertical bar  \|
%   Right brace   \}     Tilde         \~}
%
% \changes{v1.0}{2002/08/02}{First public release of
%                            \textsf{\mbox{uw-ece-workreport}}.}
% \changes{v1.1}{2003/01/11}{Minor bug fixes.}
% \changes{v2.0}{2003/04/26}{First \textsf{docstrip} release.}
% \changes{v2.0}{2003/04/21}{Renamed the class to \textsf{uw-wkrpt}.}
%
%^^A Define hanging footnotes.
% \makeatletter
% \let\footnote@rig\footnote
% \newlength{\footnoteh@ngindent}
% \renewcommand{\footnote}[1]{^^A
%   \setlength{\footnoteh@ngindent}{\parindent}^^A
%   \footnote@rig{\setlength{\hangindent}{\footnoteh@ngindent}#1}}
% \makeatother
%
%^^A This doesn't work
%^^A \GetFileInfo{uw-wkrpt.cls}
%
% \title{The \textsf{uw-wkrpt} document class\thanks{Version \fileversion, 
%                                                    last revised \filedate}}
% \author{Simon Law\thanks{\textsf{sfllaw@engmail.uwaterloo.ca}}}
% \def\today{\number\day \space\ifcase\month\or
%   January\or February\or March\or April\or May\or June\or
%   July\or August\or September\or October\or November\or December\fi
%   \space\number\year}
% \maketitle
% \begin{multicols}{2}
% \tableofcontents
% \end{multicols}
%
% \section{Introduction}
%
% At the University of Waterloo,^^A
% \footnote{\url{http://www.uwaterloo.ca/}}
% thousands of undergraduate students participate
% in the co-operative education program: a partnership between the
% University and businesses world-wide to provide first-hand experience
% for students.
%
% As part of this program, students write work reports---both to enrich
% their own literary skills, and also to provide employers with
% research that is professional, analytical, and useful.
%
% \section{Justification}
%
% The Co-operative Education and Career Services (CECS) department
% mandates certain formatting and stylistic conventions.  In addition,
% each department may impose their own conventions above and beyond
% the general CECS conventions.  Work reports are checked to ensure
% that they conform.
%
% Human beings are fallable, however, and are liable to misinterpret
% the required conventions.  Formatting a document according to fixed
% rules is something a computer should be apt at doing.  Indeed, this
% \LaTeX{} document class implements the formatting so that is done
% automatically.
%
% The inquiring mind may wonder, ``why choose \LaTeX{}?''  We must
% consider that it is not a common application with which undergraduate
% students are familiar.  Indeed, a word processor is more comfortable
% to most students.  However, implementing the requirements of each
% style in a word processor is far from simple.  Templates and styles
% are available to the student, but they are neither transparent to use
% nor easy to implement.  \LaTeX{} is a simple, macro based language
% that can format text without great user effort.  It can parse plain
% text with some sparse semantic tags to provide a decent report.  Since
% it is coupled with the world-renowned \TeX{} typesetting engine, the
% resulting report is \ae{}sthetically pleasing and typeset tastefully.
%
% A simple \LaTeX{} document can be constructed
% easily~\cite{ref:lamport}, with the knowledge
% of just a few commands.  In the example on the following page, it is 
% plain to see that the majority of the document is entered as
% plain text.
% \begingroup\center
% \begin{minipage}[c]{\ExampleWidth-5mm}
%^^A We must manually hint the hyphenation here, as the English
%^^A hyphenation module knows nothing about Latin.
%   Lorem ip\-sum do\-lor sit amet, con\-sec\-te\-tur 
%   adipi\-sicing elit, sed do eius\-mod tem\-por 
%   in\-ci\-di\-dunt ut labore et do\-lore magna ali\-qua. 
%   Ut enim ad mi\-nim ven\-iam, quis nos\-trud 
%   exer\-cita\-tion ullam\-co labo\-ris ni\-si ut 
%   ali\-quip ex ea commodo con\-se\-quat.
% \end{minipage}
% \begin{minipage}{\textwidth-\ExampleWidth}
%   \fvset{xrightmargin=0pt}
%   \begin{Verbatim}
%     \documentclass{article}
%     \begin{document}
%
%     Lorem ipsum dolor sit amet, consectetur 
%     adipisicing elit, sed do eiusmod tempor 
%     incididunt ut labore et dolore magna aliqua. 
%     Ut enim ad minim veniam, quis nostrud 
%     exercitation ullamco laboris nisi ut 
%     aliquip ex ea commodo consequat.
%
%     \end{document}
%   \end{Verbatim}
% \end{minipage}
% \endcenter\endgroup
% \noindent
% Each macro is prefixed by a backslash, and is followed by an
% alphabetic identifier.  Parameters to these macros are encased within
% curly braces.  On line 1, we declare that this document uses the
% \textsf{article} document class, which determines certain formatting
% options.  The rest of the document is created with a |\begin{document}|
% command, and finished with an |\end{document}|.
%
% Although the layout of the final text is not immediately obvious
% from the input that is keyed in, the input language is rather legible.
% One can argue that using a plain text interface allows the author
% to concentrate on the content of his message, and not the formatting.
% Since the computer does most of the formatting work, it is only
% necessary to proofread the document and tweak minor details.
%
% \section{A simple document}
%
% Sample documents that show the recommended layout are available.
% You can use these samples as a basis for your own report by removing
% the generic text, and replacing it with your own.  As well, they
% provide examples for how to typeset common forms.
% These documents are stored as 
% \texttt{uw-wkrpt-\textsl{faculty}.tex}, where
% \texttt{\textsl{faculty}} is one of:
% \begin{description}
% \item[\texttt{ece}] for Electrical and Computer Engineering (E\&CE) 
%                     students, this implements the ``E\&CE work term
%                     report guidelines''~\cite{ref:ecewrg};
% \item[\texttt{math}] for Mathematics (Math) students, this implements
%                      the ``Faculty of mathematics work report 
%                      guidelines''~\cite{ref:mwrg};
% \item[\texttt{se}] for Software Engineering (SE) students, this
%                    implements the ``Software engineering work report
%                    guidelines''~\cite{ref:sewrg}; or
% \item[\texttt{cecs}] for those students without special guidelines,
%                      see section \ref{sec:documentclass}
% \end{description}
%
% \subsection{The document class} \label{sec:documentclass}
% 
% \begin{DescribeEnv}{\textsf{uw-wkrpt}}
% Every document needs to have a document class, so it must be
% specified.  The simplest work report format is the one required by 
% CECS and specified by Chapter 9 of the ``Co-operative education
% student reference manual.''~(CESRM)~\cite{ref:cesrm}.  These
% guidelines are used by the majority of programs and can be used like
% so:
% \begin{verbatim}\documentclass{uw-wkrpt}\end{verbatim}
%
% However, some programs have their own special requirements.  Although 
% there are a number of such programs, I have only implemented the
% guidelines for E\&CE~\cite{ref:ecewrg}, Math~\cite{ref:mwrg}, and 
% SE~\cite{ref:sewrg}.  To specify these special
% requirements, we provide an optional argument to the |\documentclass|
% command.  For example, a Math student would use:
% \begin{verbatim}\documentclass[math]{uw-wkrpt}\end{verbatim}
% Notice how |[math]| is enclosed by square brackets.  Other valid
% options are |[ece]| and |[se]|.
%
% Note that no text may appear before the |\documentclass| command.
% \end{DescribeEnv}
%
% \subsection{The preamble} \label{sec:preamble}
%
% \newsavebox{\missingbegindocument}
% \begin{lrbox}{\missingbegindocument}^^A
% \verb'! LaTeX Error: Missing \begin{document}.'^^A
% \end{lrbox}
% Between the |\documentclass{uw-wkrpt}| command and the 
% |\begin{document}| command is the section known as the preamble.
% No text may occur here,^^A
% \footnote{In fact, if plain text does get put in the
% preamble \LaTeX{} will complain with the error: 
% \usebox{\missingbegindocument}}
% but commands to set initial values and options
% are declared at this point.
%
% \subsubsection{Mandatory values} \label{sec:mandvalues}
%
% The following commands define initial values that must be set.
% These values are used to typeset the title page (see Section
% \ref{sec:document},) and the letter of submittal (see Section 
% \ref{sec:prelim}.)
%
% \begin{DescribeMacro}{\title}
% The |\title|\marg{text} command defines the work report's title.  
% This will be capitalised on the title page, and included in the 
% letter of submittal.  
%
% This command is analogous to the standard \LaTeXe{} command.
% \end{DescribeMacro}
%
% \begin{DescribeMacro}{\author}
% The |\author|\marg{text} command defines the author's name.  
%
% This command is analogous to the standard \LaTeXe{} command.
% \end{DescribeMacro}
%
% \begin{DescribeMacro}{\uwid}
% The |\uwid|\marg{text} command defines the author's student 
% identification number.
% \end{DescribeMacro}
%
% \begin{DescribeMacro}{\signature}
% The |\signature|\marg{text} command defines the name of the PNG file
% with the student's signature. Only required for SE reports.
% \end{DescribeMacro}
%
% \begin{DescribeMacro}{\address}
% The |\address|\marg{text} command defines the author's home address.
% Since an address can span multiple lines, each line is separated with 
% a |\\*| command.
% \begin{verbatim}\address{200 University Ave. W.,\\*
%         Waterloo, ON\ \ N2L 3G1}\end{verbatim}
% Also note the use of \verb*|\ \ | ^^A*
% to force a double-space between the province and the postal code.
% \end{DescribeMacro}
%
% \begin{DescribeMacro}{\employer}
% The |\employer|\marg{text} command defines the employer's name.  
% Typically, this will be the company's business name.
% \end{DescribeMacro}
% 
% \begin{DescribeMacro}{\employeraddress}
% The |\employeraddress|\marg{text} command defines the employer's 
% short address, which should merely be the name of the city and province.
% For example, if the employer is located in Montr\'eal, Qu\'ebec:
% \begin{verbatim}\employeraddress{Montr\'eal, QC}\end{verbatim}
% or in New York, New York, USA:
% \begin{verbatim}\employeraddress{New York, NY}\end{verbatim}
% or in London, England:
% \begin{verbatim}\employeraddress{London, UK}\end{verbatim}
% \end{DescribeMacro}
%
% \begin{DescribeMacro}{\school}
% The |\school|\marg{text} command defines the name of the school 
% the author attends.  This should be
% \begin{verbatim}\school{University of Waterloo}\end{verbatim}
% for most students.
% \end{DescribeMacro}
%
% \begin{DescribeMacro}{\faculty}
% The |\faculty|\marg{text} command defines the faculty or program 
% the author is in.
% \end{DescribeMacro}
%
% \begin{DescribeMacro}{\userid}
% The |\userid|\marg{text} command defines the author's student id (username).
% \end{DescribeMacro}
% \changes{v2.8}{2012/04/24}{Add \textbackslash userid command.}
%
% \begin{DescribeMacro}{\email}
% The |\email|\marg{text} command defines the author's e-mail address.
% \end{DescribeMacro}
%
% \begin{DescribeMacro}{\term}
% The |\term|\marg{text} command defines the previous academic term 
% the author was enrolled in.  For instance, if the author has only 
% finished one school term, (\emph{i.e.} she is in stream four), then 
% she would use
% \begin{verbatim}\term{1A}\end{verbatim}
% because she last attended school in her 1A term.
% \end{DescribeMacro}
%
% \begin{DescribeMacro}{\program}
% The |\program|\marg{text} command defines the author's current
% program.  A student in Computer Science would write
% \begin{verbatim}\program{Computer Science}\end{verbatim}
% \end{DescribeMacro}
%
% \begin{DescribeMacro}{\chair}
% The |\chair|\marg{text} command defines the very important person to
% whom your letter of submittal is submitted.
% From Section 9.9.1 of the CESRM~\cite{ref:cesrm}:
% \begin{quote}
%   If this is your first report (except if you are in Arts, Math, AHS, 
%   Geography or Science) address your letter to Mr. B. Lumsden, 
%   Director, Co-operative Education \& Career Services.  If it is not 
%   your first report or if you are in Arts, Math, Geography or Science,
%   direct your letter to the Department Chair. If you are in AHS, your
%   letter should be addressed to the Associate Dean of your faculty.^^A
%   \footnote{This statement was quoted on 24 April 2003.}
% \end{quote}
% In Software Engineering, the letter of submittal should be addressed to
% the Director of Software Engineering.
% \end{DescribeMacro}
%
% \begin{DescribeMacro}{\chairaddress}
% The |\chairaddress|\marg{text} command defines the address to which
% you will send your report.  Like the |\address| command, you should
% break lines appropriately.
% \end{DescribeMacro}
%
% \subsubsection{Optional values} \label{sec:optvalues}
%
% Some commands are completely optional and do not have to be included
% in the preamble.
%
% \begin{DescribeMacro}{\date}
% The |\date|\marg{text} command defines an arbitrary date for the title
% page and  the letter of submittal.  By default, today's date is used.  
%
% This is analogous to the standard \LaTeXe{} command.
% \end{DescribeMacro}
%
% \begin{DescribeMacro}{\confidential}
% The |\confidential|\marg{text} command defines the confidentiality of
% the report.  Most reports do not require this command.  Refer to
% Section 9.7 of the CESRM~\cite{ref:cesrm} for more information.
% As an example, if the report is rated as ``Confidential-1'':
% \begin{verbatim}\confidential{Confidential-1}\end{verbatim}
% 
% Please be aware that there are certain restrictions for confidential
% reports.  You must speak with your field co-ordinator or faculty
% before undertaking a confidential report.  Most confidential reports
% are not marked until the following term.  If you work for certain
% corporations, your work report cannot be confidential.  If you are in
% certain faculties, your work report cannot be confidential.  For more
% information, see section 9.7 of the CESRM~\cite{ref:cesrm} and Section
% 5 of the E\&CE~\cite{ref:ecewrg} and SE~\cite{ref:sewrg} guidelines.
%
% There are several levels of confidentiality:
% \begin{description}
% \item[Not confidential] These reports can be reviewed and evaluated by 
%                         one or more markers.
% \item[Confidential-1] These reports must be stored safely, and may only
%                       be evaluated by one marker.  No duplicates may be
%                       made.
% \item[Confidential-2] One particular aspect of the report may be subject
%                       to a non-disclosure agreement.  This must be
%                       negotiated between the employer and a particular 
%                       marker.
% \item[Confidential-3] Confidential data contained in the report has been 
%                       altered to permit disclosure.
% \item[Confidential-4] The report cannot leave the employer and must be
%                       evaluated by a fellow employee.
% \item[SE-confidential] This is the only level of confidentiality for
%                        Software Engineering reports. The report is placed in
%                        an envelope marked ``SE-confidential'', and is treated
%                        with care by the markers. No duplicates may be made.
% \end{description}
% Confidential reports are not eligible for an Outstanding grade.
% For a detailed discussion of the levels of confidentiality, see
% ``Confidential work term reports''~\cite{ref:confwtr}.
% \end{DescribeMacro}
%
% \subsubsection{Accessing values}
%
% \begin{DescribeMacro}{\theauthor}
% \begin{DescribeMacro}{\thetitle}
% \begin{DescribeMacro}{\theuwid}
% \begin{DescribeMacro}{\thesignature}
% \begin{DescribeMacro}{\theaddress}
% \begin{DescribeMacro}{\theemployer}
% \begin{DescribeMacro}{\theemployeraddress}
% \begin{DescribeMacro}{\theschool}
% \begin{DescribeMacro}{\thefaculty}
% Each of these commands reproduce the text defined by the respective
% command defined in the previous sections.  Although these macros
% can be used anywhere in the report, they are used primarily in the 
% \textbf{letter} environment, see Section \ref{sec:prelim}.
% \end{DescribeMacro}
% \end{DescribeMacro}
% \end{DescribeMacro}
% \end{DescribeMacro}
% \end{DescribeMacro}
% \end{DescribeMacro}
% \end{DescribeMacro}
% \end{DescribeMacro}
% \end{DescribeMacro}
% \par^^A Insert a paragraph here to flush the floats.
% \begin{DescribeMacro}{\theuserid}
% \begin{DescribeMacro}{\theemail}
% \begin{DescribeMacro}{\theterm}
% \begin{DescribeMacro}{\theprogram}
% \begin{DescribeMacro}{\thechair}
% \begin{DescribeMacro}{\thechairaddress}
% \begin{DescribeMacro}{\thedate}
% \begin{DescribeMacro}{\theconfidential}
% Here is a more comprehensive example:
% \begingroup\center
% \begin{minipage}[c]{\ExampleWidth-5mm}
%   Hello, my name is J. Doe, and the title
%   of my report is ``My first work report.''
% \end{minipage}
% \begin{minipage}{\textwidth-\ExampleWidth}
%   \fvset{xrightmargin=0pt}
%   \begin{Verbatim}
%     \documentclass{uw-wkrpt}
%     \title{My first work report}
%     \author{J. Doe}
%     % ...more definitions ...
%     \begin{document}
%
%     Hello, my name is \theauthor, and the title
%     of my report is ``\thetitle.''
%
%     \end{document}
%   \end{Verbatim}
% \end{minipage}
% \endcenter\endgroup
% \end{DescribeMacro}
% \end{DescribeMacro}
% \end{DescribeMacro}
% \end{DescribeMacro}
% \end{DescribeMacro}
% \end{DescribeMacro}
% \end{DescribeMacro}
% \end{DescribeMacro}
%
% \subsection{The document} \label{sec:document}
%
% \begin{DescribeEnv}{document}
% Any text within the |\begin{document}| and |\end{document}| commands
% are said to be within the \textbf{document} environment.  This text
% will be typeset into the final output, and any text after the
% environment will be ignored.
% \end{DescribeEnv}
%
% \begin{DescribeMacro}{\maketitle}
% To create the title page, the |\maketitle| command is used.  This
% command should be invoked before any other text.  All the necessary
% information is contained upon this page.  A clear cover should be used
% to let this show through.
%
% This command is analogous to the standard \LaTeXe{} command.
% \end{DescribeMacro}
%
% \subsubsection{Preliminary pages} \label{sec:prelim}
%
% \begin{DescribeMacro}{\frontmatter}
% The |\frontmatter| command is used to tell \LaTeX{} that the next
% sections should be typeset as preliminary pages.  This typically
% involves lower-case roman page numbers.
%
% This command is analogous to the standard \LaTeXe{} command in the 
% \textsf{book} document class.
% \end{DescribeMacro}
%
% \begin{DescribeEnv}{letter}
% The \textbf{letter} environment does most of the difficult work
% involved in writing the letter of submittal.  When |\begin{letter}| is
% invoked, the headings and salutations are laid out.  On the next line,
% the body of the message should be entered.  The environment is closed
% with the |\end{letter}| command, which generates the boilerplate
% disclaimer required by the guidelines, and generates the signature
% block.
%
% The \textbf{letter} environment is able to get the information
% required to generate the address blocks, the date, the salutation and
% the signature because this information was defined in the preamble, see
% section \ref{sec:preamble}.
%
% The body of the report is required to contain certain information.
% According to Section 9.9.1 of the CESRM~\cite{ref:cesrm}, this includes:
% \begin{itemize}
% \item report title (use |\thetitle|) 
% \item report number (first, second, etc.)
% \item employer (use |\theemployer|)
% \item previous academic term (use |\theterm|)
% \item supervisor(s)
% \item department(s) worked for
% \item main activity of employer and department
% \item purpose of report
% \item acknowledgements and explanation of assistance
% \item statement of confidentiality, if required
% \end{itemize}
%
% Section 3.3 of the Math~\cite{ref:mwrg} guidelines also require that 
% you include:
% \begin{itemize}
% \item your role in the company
% \item brief description of your duties
% \end{itemize}
% As well, you must also left-justify your letter.  Although the Math 
% department allows for memorandums of submittal, I do not support their 
% creation.
%
% Section 2 of the E\&CE~\cite{ref:ecewrg} and SE~\cite{ref:sewrg}
% guidelines also  require that you:
% \begin{itemize}
% \item state who the report was written for
% \end{itemize}
%
% This environment is analogous to the standard \LaTeXe{} environment
% in the \textsf{letter} document class.
% \end{DescribeEnv}
%
% \begin{DescribeMacro}{\section}
% The |\section|\oarg{short}\marg{text} command is set to suppress any
% section numbering in the preliminary pages.  The \meta{text} argument
% specifies the section heading, and \meta{short} specifies the optional
% short heading for inclusion in the ``Table of Contents''.
%
% Unlike |\section*|\marg{text}, these sections are mentioned
% in the Table of Contents.  While Section 9.9.1 of the
% CESRM~\cite{ref:cesrm} states
% that preliminary pages do not appear in the Table of Contents, this is
% not an issue since their are nor proper |\section|s in a CESRM report.
% If there is a ``Summary'' section, it appears in the Body.  For a Math
% report~\cite{ref:mwrg}, the ``Summary'' is the only section in the 
% preliminary pages,
% and it should be listed in the Table of Contents.  For 
% E\&CE~\cite{ref:ecewrg} and SE~\cite{ref:sewrg} reports, all 
% preliminary sections are listed.
%
% Section 9.9.3 of the CESRM recommends that section numbers appear only
% in the body of the report, see Section \ref{sec:body}.  This
% recommendation becomes a requirement in other programs.
%
% As well, each |\section| is printed on a separate page.  This is
% implied by the CESRM, and required for other programs.  Since it does
% not hurt to put them on separate pages, it is always done.
%
% This command is analogous to the standard \LaTeXe{} command.
% \end{DescribeMacro}
%
% \begin{DescribeMacro}{\tableofcontents}
% \begin{DescribeMacro}{\listoffigures}
% \begin{DescribeMacro}{\listoftables}
% These commands generate a ``Table of Contents'', ``List of Figures''
% and ``List of Tables'' respectively.  Each table is on a separate
% page, and contains the appropriate list.
%
% Following Section 9.9.1 of the CESRM~\cite{ref:cesrm}, the Table of 
% Contents lists all sections, and subsections of a report.  Each entry 
% is connected by dotted tab leading to the page number, which is 
% right-aligned.
%
% The ``List of Figures'' and ``List of Tables'' are not considered
% sections, and are in included in the ``Table of Contents.''
% For E\&CE~\cite{ref:ecewrg} and SE~\cite{ref:sewrg} reports, however, 
% they are considered sections and are listed.
%
% These commands are analogous to the standard \LaTeXe{} commands.
% \end{DescribeMacro}
% \end{DescribeMacro}
% \end{DescribeMacro}
%
% \subsubsection{The body} \label{sec:body}
%
% \begin{DescribeMacro}{\mainmatter}
% The |\mainmatter| command is used to indicate the body of the report.
% This turns section numbering back on, and causes an arabic page number to
% appear on each page.
%
% This command is analogous to the standard \LaTeXe{} command.
% \end{DescribeMacro}
%
% \begin{DescribeMacro}{\section}
% \begin{DescribeMacro}{\subsection}
% \begin{DescribeMacro}{\subsubsection}
% The sectioning commands here will now provide numbered sections,
% labelled with the appropriate heading.
% \begingroup\center
% \newcounter{oldsection}
% \setcounter{oldsection}{\value{section}}
% \newcounter{oldsubsection}
% \setcounter{oldsubsection}{\value{subsection}}
% \newcounter{oldsubsubsection}
% \setcounter{oldsubsubsection}{\value{subsubsection}}
% \setcounter{section}{0}
% \renewcommand{\addcontentsline}[3]{}
% \begin{minipage}[c]{\ExampleWidth-5mm}
%   \section{Primary}
%   \subsection{Primier}
%   \subsubsection{Primo}
%   \section{Secondary}
%   \subsection{Deuxi\`eme}
%   \subsubsection{Secundo}
% \end{minipage}
% \setcounter{section}{\value{oldsection}}
% \setcounter{subsection}{\value{oldsubsection}}
% \setcounter{subsubsection}{\value{oldsubsubsection}}
% \begin{minipage}{\textwidth-\ExampleWidth}
%   \fvset{xrightmargin=0pt}
%   \begin{Verbatim}
%     \begin{document}
%
%     \section{Primary}
%     \subsection{Primier}
%     \subsubsection{Primo}
%     \section{Secondary}
%     \subsection{Deuxi\`eme}
%     \subsubsection{Secundo}
%
%     \end{document}
%   \end{Verbatim}
% \end{minipage}
% \endcenter\endgroup
%
% These commands are analogous to the standard \LaTeXe{} commands.
% \end{DescribeMacro}
% \end{DescribeMacro}
% \end{DescribeMacro}
%
% \begin{DescribeEnv}{figure}
% \begin{DescribeEnv}{table}
% The \textbf{figure} and \textbf{table} environments are used to create
% a ``float'' which encapsulates a graphic or a \textbf{tabular}
% environment, respectively.
%
% By defaults, floats try to place themselves at the top of the current 
% page, however, Section 9.9.3 of the CESRM~\cite{ref:cesrm} suggests 
% that figures and tables appear only after they are referenced in the 
% text.  Other programs require this behaviour.  Therefore, a float will 
% now try to place itself immediately after the |\begin{figure}| or 
% |\begin{table}| command.  If this is not possible, the float tries to 
% place itself at the end of the current page.  If this is still not 
% possible, it will center itself on a dedicated page.
%
% Figures must have their captions below, and tables must have their 
% captions on top.  Section 9.9.3 of the CESRM~\cite{ref:cesrm}.
% shows some examples.
%
% These environments are analogous to the standard \LaTeXe{}
% environments.
% \end{DescribeEnv}
% \end{DescribeEnv}
%
% \subsubsection{Back matter}
%
% \begin{DescribeMacro}{\appendix}
% The |\appendix| command indicates that section numbers should now be
% reset, and in uppercase letters.  That is to say that the first
% |\section| command will be listed as appendix ``A''.
%
% Although appendices appear in the back matter, this command should be
% issues before the |\backmatter| command.
%
% This command is analogous to the standard \LaTeXe{} command.
% \end{DescribeMacro}
%
% \begin{DescribeMacro}{\backmatter}
% The |\backmatter| command is used to indicate the back of the report.
% This turns section numbering off once more.
%
% This command is analogous to the standard \LaTeXe{} command.
% \end{DescribeMacro}
%
% \begin{DescribeMacro}{\bibliography}
% The |\bibliography| command is used to insert the bibliography, or
% ``References'' section.  This should come after the |\backmatter|
% command, and refer to a B\textsc{ib}\hspace{-0.1ex}\TeX{} database.
%
% This command is analogous to the standard \LaTeXe{} command.
% \end{DescribeMacro}
%
% \section{Implementation}
% \iffalse
%<*class>
% \fi
%
% To parse the required arguments, the \textsf{ifthen} package is loaded.
% This way, the standard \LaTeX{} facilities can be used instead of the
% \TeX{} |\if| primitives.
%    \begin{macrocode}
\RequirePackage{ifthen}
%    \end{macrocode}
%
% \subsection{Parsing options} \label{sec:parseopts}
% Since this is a document class, the first thing to do is parse out the
% options that were passed in.  To specify which program this work
% report is written for, the author passes either \texttt{math},
% \texttt{ece} or \texttt{se} options.  By default, we use the CESRM
% guidelines~\cite{ref:cesrm}.
%
% So, the options are declared, and boolean flags of the form
% |uwwkrpt@|\meta{program} are declared.
%    \begin{macrocode}
\newif\ifuwwkrpt@math \uwwkrpt@mathfalse
\DeclareOption{math}{%
  \uwwkrpt@mathtrue
  \write10{([math] Mathematics report)}}
\newif\ifuwwkrpt@ece \uwwkrpt@ecefalse
\DeclareOption{ece}{%
  \uwwkrpt@ecetrue
  \write10{([ece] Electrical and Computer Engineering report)}}
\newif\ifuwwkrpt@se \uwwkrpt@sefalse
\DeclareOption{se}{%
  \uwwkrpt@setrue
  \write10{([se] Software Engineering report)}}
%    \end{macrocode}
%
% \changes{v2.0}{2003/04/26}{Select between different programs'
%                            guidelines.}
%
% Work reports must always be set in 12 pt.\ type.  Warn the author if
% he specifies smaller type, and use 12 pt.\ nevertheless.
% 
% E\&CE guidelines requre work reports to be set in 11 pt.\ type.
% 
%    \begin{macrocode}
\ifthenelse{\boolean{uwwkrpt@ece}}{%
  \def\uwwkrpt@textsize{11pt}}{%
  \def\uwwkrpt@textsize{12pt}}

\DeclareOption{10pt}{\ClassWarning{uw-wkrpt}{%
  You requested a 10pt font but reports must be \uwwkrpt@textsize}}

\ifthenelse{\boolean{uwwkrpt@ece}}{%
  \DeclareOption{12pt}{\ClassWarning{uw-wkrpt}{%
    You requested a 12pt font but reports must be \uwwkrpt@textsize}}}{%
  \DeclareOption{11pt}{\ClassWarning{uw-wkrpt}{%
    You requested a 11pt font but reports must be \uwwkrpt@textsize}}}
%    \end{macrocode}
%
% \changes{v2.0}{2003/04/25}{Enforce 12 pt.\ type.}
% \changes{v2.7}{2011/09/16}{Enforce 11 pt.\ type for ECE reports.}
% \changes{v2.8}{2012/04/24}{Fix bug so letter format options work. Adjust the
%                            spacing for `blockletter' format.}
% \changes{v2.8.1}{2012/04/25}{Remove `blockletter' and `modifiedletter'
%                              options.}
%
% All of the options specific to this class are declared.  The rest of
% the options will be passed to the standard \LaTeXe{} \textsf{article}
% document class, the options processed, and the \textsf{article} class 
% loaded.  
%    \begin{macrocode}
\DeclareOption*{\PassOptionsToClass {\CurrentOption}{article}}
\ProcessOptions
\LoadClass[titlepage,\uwwkrpt@textsize]{article}
%    \end{macrocode}
%
% \subsection{Page margins}
%
% The standard North American paper size is U.S. letter, sized 8.5 by 11
% inches.  This is the default.
%
% The left and right margins will be set to 1.5 inches wide; the top
% and bottom margins will be set to 1.0 inches wide.  This is required
% by Section 9.8.5 of the CESRM~\cite{ref:cesrm}.
%
% For ECE work reports, the left margin will be set to 1.5 inches wide;
% the other margins will be 1.0 inches wide.
%
% For SE work reports, all margins are set to 1.0 inches wide.
%    \begin{macrocode}
\newlength{\marginl}
\newlength{\marginr}
\newlength{\margintb}

\setlength{\marginl}{1.5in}
\setlength{\marginr}{1.5in}
\setlength{\margintb}{1in}

\ifthenelse{\boolean{uwwkrpt@ece} \or \boolean{uwwkrpt@se}}
  {\setlength{\marginr}{1in}}{}

\ifthenelse{\boolean{uwwkrpt@se}}
  {\setlength{\marginl}{1in}}{}

\RequirePackage[top=\margintb, bottom=\margintb, left=\marginl, right=\marginr]{geometry}
%    \end{macrocode}
% \changes{v2.0}{2003/04/21}{Set margins correctly.}
% \changes{v2.2}{2003/05/05}{Set top and bottom margins to 1 inch.}
% \changes{v2.7}{2011/09/16}{Set margins correctly for E\&CE reports.}
% \changes{v2.8}{2012/04/24}{Set margins correctly for SE reports.}
%
% \subsection{Spacing}
%
% Spacing is rather important in this document, as there are several
% requirements for line spacing.
%
% To facilitate changing from single-spaced to one-and-half-spaced
% or double-spaced throughout the document, the \textsf{setspace} package 
% is loaded.  See Section 9.8.5 of the CESRM~\cite{ref:cesrm}.
%
% For Mathematics students, their reports must be double-spaced (See Section
% 3.1 of the Math guidelines~\cite{ref:mwrg}.)  For
% Software Engineering and ECE students, their reports must be one-and-half-spaced
% (See Section 2 of the SE guidelines~\cite{ref:sewrg}.)
% \changes{v2.4}{2003/05/10}{Fixed one-and-half-spacing vs. double-spacing.}
%    \begin{macrocode}
\RequirePackage{setspace}
\newcommand{\uwwkrpt@spacing}{\doublespacing}
\ifthenelse{\boolean{uwwkrpt@se} \or \boolean{uwwkrpt@ece}}
  {\renewcommand{\uwwkrpt@spacing}{\onehalfspacing}}{}
%    \end{macrocode}
%
% Each paragraph must be followed by a blank line, see Section 9.8.5 of
% the CESRM~\cite{ref:cesrm}.  Instead of introducing a completely blank 
% line, which is hideous due to spacing issues, we space each paragraph 
% apart by an ex-height.\footnote{This is the height of the lower-case
% letter `x'.}
%    \begin{macrocode}
\newlength{\uwwkrpt@parskip}
\ifthenelse{\boolean{uwwkrpt@se}}
  {\setlength{\uwwkrpt@parskip}{1em}}
  {\setlength{\uwwkrpt@parskip}{1ex}}
\setlength{\parskip}{\uwwkrpt@parskip}
\ifthenelse{\boolean{uwwkrpt@se}}{\setlength{\parindent}{0.4in}}{}
%    \end{macrocode}
% \changes{v2.0}{2003/04/21}{Set paragraph spacing correctly.}
% \changes{v2.1}{2003/05/02}{Set a standard paragraph skip.}
% \changes{v2.8}{2012/04/24}{Increase the paragraph indent for SE reports.}
% \changes{v2.8.3}{2012/09/16}{Increase the paragraph skip for SE reports.}
%
% \subsection{Miscellaneous packages}
%
% The \textsf{url} package is also loaded, since it breaks
% URLs\footnote{Uniform Resource Locators} and URIs\footnote{Uniforce
% Resource Identifiers} across lines.  However, the default typewriter
% font is unappealing in plain text, so it has been switched to
% sans-serif.
%    \begin{macrocode}
\RequirePackage{url}
\urlstyle{sf}
%    \end{macrocode}
%
% \subsection{Manditory and optional values}
%
% We override the standard \LaTeX{} commands |\title|, |\author|, and
% |\date|.  
%
% \begin{macro}{\title}
% The title must be defined, and is therefore enforced.  See Section
% \ref{sec:mandvalues}.
%    \begin{macrocode}
\renewcommand{\title}[1]{%
  \renewcommand{\@title}{#1}%
  \renewcommand{\@@title}{#1}}
\newcommand{\@@title}{\ClassError{uw-wkrpt}%
  {No \noexpand\title given}{}}
%    \end{macrocode}
% \changes{v2.0}{2003/04/26}{\LaTeXe{} style definitions.}
% \end{macro}
%
% \begin{macro}{\author}
% The author must be defined, and is therefore enforced.  See Section
% \ref{sec:mandvalues}.
%    \begin{macrocode}
\renewcommand{\author}[1]{%
  \renewcommand{\@author}{#1}%
  \renewcommand{\@@author}{#1}}
\newcommand{\@@author}{\ClassError{uw-wkrpt}%
  {No \noexpand\author given}{}}
%    \end{macrocode}
% \changes{v2.0}{2003/04/26}{\LaTeXe{} style definitions.}
% \end{macro}
%
% \begin{macro}{\date}
% The date defaults to today's date.  This is still an optional command,
% see Section \ref{sec:optvalues}.
%    \begin{macrocode}
\renewcommand{\date}[1]{%
  \renewcommand{\@date}{#1}%
  \renewcommand{\@@date}{#1}}
\newcommand{\@@date}{\today}
%    \end{macrocode}
% \changes{v2.0}{2003/04/26}{\LaTeXe{} style definitions.}
% \end{macro}
%
% \begin{macro}{\uwid}
% \begin{macro}{\signature}
% \begin{macro}{\address}
% \begin{macro}{\employer}
% \begin{macro}{\employeraddress}
% \begin{macro}{\school}
% \begin{macro}{\faculty}
% \begin{macro}{\userid}
% \begin{macro}{\email}
% \begin{macro}{\term}
% \begin{macro}{\program}
% \begin{macro}{\chair}
% \begin{macro}{\chairaddress}
% New variables which are defined.  These, like the ones
% above, are used to construct the title page.  As well, they can be
% used to construct the letter of submittal.
%
% The following are manditory values, see Section \ref{sec:mandvalues}.
%    \begin{macrocode}
\newcommand{\uwid}[1]{\renewcommand{\@uwid}{#1}}
  \newcommand{\@uwid}{\ClassError{uw-wkrpt}%
    {No \noexpand\uwid given}{}}
\newcommand{\signature}[1]{\renewcommand{\@signature}{#1}}
  \newcommand{\@signature}{\ClassError{uw-wkrpt}%
    {No \noexpand\signature given}{}}
\newcommand{\address}[1]{\renewcommand{\@address}{#1}}
  \newcommand{\@address}{\ClassError{uw-wkrpt}%
    {No \noexpand\address given}{}}
\newcommand{\employer}[1]{\renewcommand{\@employer}{#1}}
  \newcommand{\@employer}{\ClassError{uw-wkrpt}%
    {No \noexpand\employer given}{}}
\newcommand{\employeraddress}[1]{\renewcommand{\@employeraddress}{#1}}
  \newcommand{\@employeraddress}{\ClassError{uw-wkrpt}%
    {No \noexpand\employeraddress given}{}}
\newcommand{\school}[1]{\renewcommand{\@school}{#1}}
  \newcommand{\@school}{\ClassError{uw-wkrpt}%
    {No \noexpand\school given}{}}
\newcommand{\faculty}[1]{\renewcommand{\@faculty}{#1}}
  \newcommand{\@faculty}{\ClassError{uw-wkrpt}%
    {No \noexpand\faculty given}{}}
\newcommand{\userid}[1]{\renewcommand{\@userid}{#1}}
  \newcommand{\@userid}{\ClassError{uw-wkrpt}%
    {No \noexpand\userid given}{}}
\newcommand{\email}[1]{\renewcommand{\@email}{#1}}
  \newcommand{\@email}{\ClassError{uw-wkrpt}%
    {No \noexpand\email given}{}}
\newcommand{\term}[1]{\renewcommand{\@term}{\textsc{\lowercase{#1}}}}
  \newcommand{\@term}{\ClassError{uw-wkrpt}%
    {No \noexpand\term given}{}}
\newcommand{\program}[1]{\renewcommand{\@program}{#1}}
  \newcommand{\@program}{\ClassError{uw-wkrpt}%
    {No \noexpand\program given}{}}
\newcommand{\chair}[1]{\renewcommand{\@chair}{#1}}
  \newcommand{\@chair}{\ClassError{uw-wkrpt}%
    {No \noexpand\chair given}{}}
\newcommand{\chairaddress}[1]{\renewcommand{\@chairaddress}{#1}}
  \newcommand{\@chairaddress}{\ClassError{uw-wkrpt}%
    {No \noexpand\chairaddress given}{}}
%    \end{macrocode}
% \changes{v2.8}{2012/04/24}{Add \textbackslash userid field.}
%
% \end{macro}
% \end{macro}
% \end{macro}
% \end{macro}
% \end{macro}
% \end{macro}
% \end{macro}
% \end{macro}
% \end{macro}
% \end{macro}
% \end{macro}
% \end{macro}
% \end{macro}
%
% \begin{macro}{\confidential}
% |\confidential| is an optional value, see Section \ref{sec:optvalues}.
% If it is empty, it will be ignored.  Since most reports are
% non-confidential, this is the default value.
%    \begin{macrocode}
\newcommand{\confidential}[1]{\renewcommand{\@confidential}{#1}}
  \newcommand{\@confidential}{}
%    \end{macrocode}
% \end{macro}
%
% \begin{macro}{\thetitle}
% \begin{macro}{\theauthor}
% \begin{macro}{\thedate}
% \begin{macro}{\theuwid}
% \begin{macro}{\thesignature}
% \begin{macro}{\theaddress}
% \begin{macro}{\theemployer}
% \begin{macro}{\theemployeraddress}
% \begin{macro}{\theschool}
% \begin{macro}{\thefaculty}
% \begin{macro}{\theuserid}
% \begin{macro}{\theemail}
% \begin{macro}{\theterm}
% \begin{macro}{\theprogram}
% \begin{macro}{\thechair}
% \begin{macro}{\thechairaddress}
% The following commands are defined to access these values of these
% new variables in case the author wishes to refer to them within the 
% document.
%    \begin{macrocode}
\newcommand{\thetitle}{\@@title}
\newcommand{\theauthor}{\@@author}
\newcommand{\theuwid}{\@uwid}
\newcommand{\thesignature}{\@signature}
\newcommand{\theaddress}{\@address}
\newcommand{\theemployer}{\@employer}
\newcommand{\theemployeraddress}{\@employeraddress}
\newcommand{\theschool}{\@school}
\newcommand{\thefaculty}{\@faculty}
\newcommand{\theuserid}{\@userid}
\newcommand{\theemail}{\@email}
\newcommand{\theterm}{\@term}
\newcommand{\theprogram}{\@program}
\newcommand{\thechair}{\@chair}
\newcommand{\thechairaddress}{\@chairaddress}
\newcommand{\thedate}{\@@date}
\newcommand{\theconfidential}{\@confidential}
%    \end{macrocode}
% \end{macro}
% \end{macro}
% \end{macro}
% \end{macro}
% \end{macro}
% \end{macro}
% \end{macro}
% \end{macro}
% \end{macro}
% \end{macro}
% \end{macro}
% \end{macro}
% \end{macro}
% \end{macro}
% \end{macro}
% \end{macro}
%
% \subsection{Title page}
%
% We require the \textsf{textcase} package to provide the
% |\MakeTextUppercase|\marg{text} command.
%    \begin{macrocode}
\RequirePackage{textcase}
%    \end{macrocode}
%
% \begin{macro}{\maketitle}
% The title page must be laid out in a certain format, for an example
% see Figure 1 of Section 9.9.1 of the CESRM~\cite{ref:cesrm}.
% There are minor changes to the Software Engineering format, but this is
% mainly a matter of personal preference.
%    \begin{macrocode}
\renewcommand{\maketitle}{%
  \begin{titlepage}
  \begin{singlespacing}
  \let\footnotesize\small
  \let\footnoterule\relax
  \let \footnote \thanks
  \begin{center}
    {\large \MakeTextUppercase{\@school} \par \@faculty}%
  \end{center}
  \null\vfill%
  \begin{center}%
    \ifthenelse{\boolean{uwwkrpt@se}}
      {\LARGE \@title \par}{\huge \MakeTextUppercase{\@title} \par}%
  \end{center}\par
  \null\vfill%
  \begin{center}%
    {\large \@employer\\ \@employeraddress\par \textit{\@confidential}}%
  \end{center}\par
  \null\vfill%
  \begin{center}{%
    \ifthenelse{\boolean{uwwkrpt@se}}{\normalsize}{\large}
    \ifthenelse{\boolean{uwwkrpt@se}}{\textbf{Prepared by}\\}{Prepared by\\}
      \begin{tabular}[t]{c}%
        \@author\\
        \ifthenelse{\boolean{uwwkrpt@se}}
          {Student ID: \@uwid\\ User ID: \@userid\\}{ID \#\@uwid\\ \@email\\}
        \@term{} \@program
      \end{tabular}\par}%
    {\ifthenelse{\not \boolean{uwwkrpt@se}}{\large}{} \@date \par}%
  \end{center}
  \@thanks
  \end{singlespacing}
  \end{titlepage}%
%    \end{macrocode}
% \changes{v2.8}{2012/04/24}{Made some SE-only tweaks.}
% After defining the title page, commands we no longer need are let go.
%    \begin{macrocode}
  \setcounter{footnote}{0}%
  \global\let\thanks\@gobble
  \global\let\maketitle\relax
  \global\let\@thanks\@empty
  \global\let\@author\@empty
  \global\let\@date\@empty
  \global\let\@title\@empty
  \global\let\title\relax
  \global\let\author\relax
  \global\let\date\relax
  \global\let\and\relax
}
%    \end{macrocode}
% \changes{v2.0}{2003/04/25}{Fixed \textbackslash thanks command.}
% \end{macro}
%
% As well, this page should have no page numbering.
%
% \subsection{Letter of submittal}
%
% \begin{environment}{letter}
% The \textbf{letter} environment simplifies the process of writing a
% letter of submittal.
%    \begin{macrocode}
\newenvironment{letter}{%
%    \end{macrocode}
%
% We turn off page numbering for the letter.  We use |\everyvbox| to
% suppress the page numbering on every page.  This is much better than
% trying to save the page numbering and then restore it.
% \changes{v2.0}{2002/04/26}{Smarter page number suppression.}
%    \begin{macrocode}
  \everyvbox={\thispagestyle{empty}}%
%    \end{macrocode}
%
% Using |\@setletterpagenum|, we decide on the logical page number for
% the letter of submittal.  For the declaration of this macro, see
% Section \ref{sec:toc}.
%    \begin{macrocode}
  \@setletterpagenum%
%    \end{macrocode}
%
% Due to reasons I cannot comprehend, Section 3.3 of the Math
% guidelines~\cite{ref:mwrg} requires that the letter of submittal be
% left-justified.
%    \begin{macrocode}
  \ifthenelse{\boolean{uwwkrpt@math}}
    {\raggedright}{}
%    \end{macrocode}
%
% We use the standard business letter format, which means paragraphs
% are not indented.
%    \begin{macrocode}
  \setlength{\parindent}{0pt}
  \setlength{\parskip}{\uwwkrpt@parskip}
%    \end{macrocode}
%
% \changes{v2.0}{2003/04/26}{New options for letter formats.}
% \changes{v2.8.1}{2012/04/25}{All letter formats are now the same.}
%
% Then, the letter is set to single-spaced, since it is not part of the
% report; but rather an insert.
%    \begin{macrocode}
  \singlespacing%
%    \end{macrocode}
%
% The header block is created.  First, the author and the author's
% address; then the current date; then the receiver of the report and
% his address; and finally the salutation.
%
% Note that for Software Engineering reports, the letter is addressed to
% the director, not the chair.
%    \begin{macrocode}
  \noindent\@@author\\\@address\par\noindent%
  \@@date \par\noindent%
  \ifthenelse{\boolean{uwwkrpt@se}}
    {\@chair, Director\\*\@chairaddress\par\noindent}%
    {\@chair, Chair\\*\@chairaddress\par\noindent}%
  Dear \@chair:%
%    \end{macrocode}
%
% Create a subject line, much like formal business letters of old. This is
% omitted from Software Engineering reports.
%    \begin{macrocode}
  \ifthenelse{\boolean{uwwkrpt@se}}
    {\par}
    {\begin{center}\textbf{Re: Submission of my work term report.}\end{center}}}
%    \end{macrocode}
%
% The author types her letter, and mentions all the things she is
% required to mention.  See Section \ref{sec:prelim} for a full list.
%
% When she is done, she ends the environment.  This triggers the
% disclaimer boilerplate, required by Section 9.9.1 of the
% CESRM~\cite{ref:cesrm}.
%    \begin{macrocode}
  {\par I hereby confirm that I have received no help, other
  than what is mentioned above, in writing this report.
%    \end{macrocode}
%
% The E\&CE~\cite{ref:ecewrg} and SE~\cite{ref:sewrg} guidelines require
% an additional boilerplate message.  I will include this boilerplate in
% the Math report, even though it does not require it.
%    \begin{macrocode}
  \ifthenelse{\boolean{uwwkrpt@ece}
              \or \boolean{uwwkrpt@math}
              \or \boolean{uwwkrpt@se}}
    {I also confirm that this report has not been previously submitted 
    for academic credit at this or any other academic institution.}
%    \end{macrocode}
%
% In other programs, the following boilerplate is required.  See Section
% 9.9.1 of the CESRM~\cite{ref:cesrm}.
%    \begin{macrocode}
    {This report was written entirely by me and has not received
    any previous academic credit at this or any other institution.}
%    \end{macrocode}
%
% The Faculty of Mathematics has a special request, since they ask
% employers to perform technical marking.  It is included after the
% legal boilerplate.  See Figure 4 of the Math guidelines~\cite{ref:mwrg}.
%    \begin{macrocode}
  \ifthenelse{\boolean{uwwkrpt@math}}{%
    \par The Faculty of Mathematics requests that you evaluate this report
    for command of topic and technical content/analysis.  Following your
    assessment, the report, together with your evaluation, will be submitted
    to the Math Undergrad Office for evaluation on campus by qualified
    work report markers.  The combined marks will determine whether the
    report will receive credit and whether it will be considered for an
    award.
    \par I would like to thank you for your assistance in preparing this
    document.}{}%
%    \end{macrocode}
%
% \changes{v2.0}{2003/04/26}{Legal boilerplate for each program.}
% \changes{v2.8}{2012/04/24}{Some more SE-only tweaks to the letter.}
% \changes{v2.8.3}{2012/09/16}{SE boilerplate should not have `further' in it.}
%
% With the legal requirements completed, the signature block can be
% generated.  The signature line is only 3 inches long and 0.3 in tall,
% but that should be sufficient for most purposes.  To satisfy Section
% 9.1 of the CESRM~\cite{ref:cesrm}, the student's name and ID are
% listed below the signature line.
%
% Software Engineering only requires the signature, student's name, and
% student's ID. No signature line is required and the signature block is
% indented. Additionally, because of electronic submissions, a PNG file
% with the student's signature can be specified, and it will be
% automatically inserted.
%    \begin{macrocode}
  \par\noindent
  \begin{minipage}{\textwidth}
  \ifthenelse{\boolean{uwwkrpt@se}}{\setlength{\parindent}{3in}}{}
  \setlength{\parskip}{\uwwkrpt@parskip}
  \vspace*{\uwwkrpt@parskip}
  \ifthenelse{\boolean{uwwkrpt@se}}{
  Sincerely,
  \vspace*{0.75em}\\
  \indent \includegraphics{\@signature}\\
  \indent \@@author\\
  \indent Student ID: \@uwid}%
  {Yours sincerely,\\*%
  \rule{0in}{0.3in}\\*{\hrule \@width 3in}%
  \noindent\@@author, \@uwid
  \par\noindent
  Encl.}%
  \end{minipage}
%    \end{macrocode}
%
% \changes{v2.2}{2003/05/02}{Keep the signature block together.}
% \changes{v2.8}{2012/04/24}{More SE-only tweaks to the signature block.}
% \changes{v2.8.3}{2012/09/16}{Add signature image for SE reports.}
%
% Now that the letter is done, we set the correct page number for pages
% that follow the letter, and then restore double-spacing.
%    \begin{macrocode}
  \@setpostletterpagenum\uwwkrpt@spacing%
  \ifthenelse{\not \boolean{uwwkrpt@se}}{\newpage}{}%
%    \end{macrocode}
%
% \changes{v2.8}{2012/04/24}{Ensure page number shows up on the Executive
%                            Summary for SE reports.}
%
% All the excess variables that were used can now be let go.
%    \begin{macrocode}
  \global\let\@@author\@empty
  \global\let\@@title\@empty
  \global\let\@@date\@empty
  \global\let\uwid\relax
  \global\let\@uwid\@empty
  \global\let\@signature\relax
  \global\let\signature\relax
  \global\let\userid\relax
  \global\let\@userid\@empty
  \global\let\email\relax
  \global\let\@email\@empty
  \global\let\employer\relax
  \global\let\@employer\@empty
  \global\let\employeraddress\relax
  \global\let\@employeraddress\@empty
  \global\let\address\relax
  \global\let\@address\@empty
  \global\let\chair\relax
  \global\let\@chair\@empty
  \global\let\chairaddress\relax
  \global\let\@chairaddress\@empty
  \global\let\school\relax
  \global\let\@school\@empty
  \global\let\faculty\relax
  \global\let\@faculty\@empty
  \global\let\term\relax
  \global\let\@term\@empty
  \global\let\program\relax
  \global\let\@program\@empty
  \global\let\confidential\relax
  \global\let\@confidential\@empty
}
%    \end{macrocode}
% \end{environment}
%
% \subsection{Formatting sections}
% We shall emulate the |\frontmatter|, |\mainmatter|, and |\backmatter|
% commands from the \textsf{book} document class.  Front matter pages 
% are numbered with roman numerals; main and back matter pages are 
% numbered with arabic numerals.  See Section 9.8.5 of the
% CESRM~\cite{ref:cesrm}.
%
% In the front and back matter, the |\section| command does not generate
% headers with section numbers, but does enter them into the Table of
% Contents.
%
% \begin{macro}{\frontmatter}
% A fresh page is started, the sections are unnumbered, and the page
% numbers are lower-case roman numberals.
%    \begin{macrocode}
\newcommand{\frontmatter}{%
  \clearpage
  \@notmainsect%
  \pagenumbering{roman}%
  \ifthenelse{\boolean{uwwkrpt@ece}}{%
  \singlespacing}{\uwwkrpt@spacing}%
}
%    \end{macrocode}
% \end{macro}
%
% \begin{macro}{\mainmatter}
% A fresh page is started, the sections are numbered, and the page
% numbers are arabic numerals.
%    \begin{macrocode}
\newcommand{\mainmatter}{%
  \clearpage
  \uwwkrpt@spacing
  \@mainsect%
  \pagenumbering{arabic}%
%    \end{macrocode}
%
% Section 3.1 of the Math guidelines~\cite{ref:mwrg} state that numbered 
% sections should not have page breaks between them.  This is why we
% restore |\section| to its original form.
%    \begin{macrocode}
  \ifthenelse{\boolean{uwwkrpt@math}}%
    {\let\section\section@rig}
%    \end{macrocode}
% \begin{macro}{\dotzero}
% \begin{macro}{\@secdotzerostart}
% \begin{macro}{\@secdotzeroend}
% The Math guidelines~\cite{ref:mwrg} imply that numbered |\section|s 
% must be of the form ``1.0'', not the default ``1''.
%
%    \begin{macrocode}
  \global\def\dotzero{}
  \global\def\@secdotzerostart##1{}
  \global\def\@secdotzeroend##1{}
  \ifthenelse{\boolean{uwwkrpt@math}}{%
    \renewcommand{\@secdotzerostart}[1]{%
      \let\quad@rig\quad
      \ifthenelse{\equal{##1}{section}}{%
        \renewcommand{\quad}{.0\quad@rig}%
        \renewcommand{\dotzero}{.0}}{\renewcommand{\dotzero}{}}
    }
    \renewcommand{\@secdotzeroend}[1]
      {\ifthenelse{\equal{##1}{section}}{\let\quad\quad@rig}}
  }{}%
%    \end{macrocode}
%
% \begin{macro}{\@appendixtitle}
%
% The E\&CE guidelines apparently want appendix titles to be ``Appendix A''.
% This is merely shown by example.  I suspect that the Software Engineering
% people may also have this, but it is also not explicit.
%
%    \begin{macrocode}
  \global\def\@appendixtitle{}
}
%    \end{macrocode}
% \end{macro}
% \changes{v2.6}{2003/05/20}{`Appendix' should prefix appendix numbers.}
% \end{macro}
% \changes{v2.3}{2003/05/09}{Numbered sections end in ``.0'' in Math.}
% \end{macro}
% \end{macro}
% \changes{v2.3}{2003/05/09}{Numbered sections don't have page breaks in Math.}
% \end{macro}
%
% \begin{macro}{\appendix}
% The page numbers turn Roman here.  The only change is that for Math,
% we eliminate the ``.0'' trailer, as ``Section A.0'' looks silly.
%    \begin{macrocode}
\let\appendix@rig\appendix
\renewcommand{\appendix}{%
  \@mainsect%
  \ifthenelse{\boolean{uwwkrpt@math}}{%
    \renewcommand{\@secdotzerostart}[1]{\renewcommand{\dotzero}{}}
    \renewcommand{\@secdotzeroend}[1]{}
    }{}%
  \ifthenelse{\boolean{uwwkrpt@ece}}
    {\renewcommand{\@appendixtitle}{Appendix }}{}
  \appendix@rig%
}
%    \end{macrocode}
% \changes{v2.3}{2003/05/09}{Numbered sections end in ``.0'' in Math.}
% \changes{v2.5}{2003/05/10}{Fix section numbering in Appendices.}
% \changes{v2.6}{2003/05/20}{`Appendix' should prefix appendix numbers.}
% \changes{v2.8}{2012/04/24}{`Appendix' prefixes appendix numbers for SE.}
% \changes{v2.8.2}{2012/09/09}{`Appendix' prefix for SE reports is moved to
%                              .tex file, where there's more control over its
%                              settings.}
% \end{macro}
%
% \begin{macro}{\backmatter}
% A fresh page is started, and the sections are unnumbered.
%    \begin{macrocode}
\newcommand{\backmatter}{%
  \clearpage
  \@notmainsect%
  \ifthenelse{\boolean{uwwkrpt@math}}%
    {\renewcommand{\section}{\clearpage\section@rig}}{}%
  \ifthenelse{\boolean{uwwkrpt@ece}}%
    {\singlespacing}%
}
%    \end{macrocode}
% \changes{v2.3}{2003/05/09}{Numbered sections don't have page breaks in Math.}
% \changes{v2.7}{2011/09/16}{Back matter is single-spaced in E\&CE.}
% \end{macro}
%
% \begin{environment}{summary}
% Much like the \textbf{abstract} environment in standard \LaTeX, the
% \textbf{summary} environment should be used for typesetting summaries.
% However, all it does is suppress the section numbers.  This
% environment should only be used by people following the
% CESRM~\cite{ref:cesrm} guidelines.  Other programs place their
% ``Executive Summary'', ``Summary'', or ``Abstract'' in different
% places.
%    \begin{macrocode}
\newenvironment{summary}
  {\@notmainsect}
  {\@mainsect}
%    \end{macrocode}
% \end{environment}
%
% \begin{macro}{\@notmainsect}
% This is the macro that turns off section numbers.  It does this by
% redefining certain functions to be much simpler, which implies that it
% no longer compensates from the prefixed number.
%
% This macro was inspired by the standard \LaTeX{} \textsf{book} class.
%    \begin{macrocode}
\newcommand{\@notmainsect}{%
  \def\@sect##1##2##3##4##5##6[##7]##8{%
    \@tempskipa ##5\relax
    \ifdim \@tempskipa>\z@
      \begingroup
        ##6{%
          \@hangfrom{\hskip ##3}%
            \interlinepenalty \@M ##8\@@par}%
      \endgroup
      \csname ##1mark\endcsname{##7}%
      \addcontentsline{toc}{##1}{##7}%
    \else
      \def\@svsechd{%
        ##6{\hskip ##3\relax
        \@svsec ##8}%
        \csname ##1mark\endcsname{##7}%
        \addcontentsline{toc}{##1}{##7}}%
    \fi
    \@xsect{##5}}%
}
%    \end{macrocode}
% \end{macro}

% \begin{macro}{\@mainsect}
% This is the macro that turns on section numbers.
% It redefines certain functions to be exactly like their standard forms.
%
% This macro was inspired by the standard \LaTeX{} \textsf{book} class.
% \begin{macrocode}
\newcommand{\@mainsect}{%
  \def\@sect##1##2##3##4##5##6[##7]##8{%
    \ifnum ##2>\c@secnumdepth
      \let\@svsec\@empty
    \else
      \refstepcounter{##1}%
      \@secdotzerostart{##1}
      \protected@edef\@svsec{\@appendixtitle\@seccntformat{##1}\relax}%
      \@secdotzeroend{##1}
    \fi
    \@tempskipa ##5\relax
    \ifdim \@tempskipa>\z@
      \begingroup
        ##6{%
          \@hangfrom{\hskip ##3\relax\@svsec}%
            \interlinepenalty \@M ##8\@@par}%
      \endgroup
      \csname ##1mark\endcsname{##7}%
      \addcontentsline{toc}{##1}{%
        \ifnum ##2>\c@secnumdepth \else
          \protect\numberline{\@appendixtitle\csname the##1\endcsname\dotzero}
          \protect\phantom{\@appendixtitle}%
        \fi
        ##7}%
    \else
      \def\@svsechd{%
        ##6{\hskip ##3\relax
        \@svsec ##8}%
        \csname ##1mark\endcsname{##7}%
        \addcontentsline{toc}{##1}{%
          \ifnum ##2>\c@secnumdepth \else
            \protect\numberline{\@appendixtitle\csname the##1\endcsname\dotzero}
            \protect\phantom{\@appendixtitle}%
          \fi
          ##7}}%
    \fi
    \@xsect{##5}}%
}
%    \end{macrocode}
% \end{macro}
%
% \begin{macro}{\section}
% Every section must start on a separate page.  Overloading the
% |\section| command ensures this.  Although the CESRM~\cite{ref:cesrm}
% implies this, Math~\cite{ref:mwrg}, and Engineering demand this, 
% see section 3.1 of the Math guidelines~\cite{ref:mwrg}; and
% SE~\cite{ref:sewrg} guidelines.
%    \begin{macrocode}
\let\section@rig\section
\ifthenelse{\not \boolean{uwwkrpt@ece}}{
  \renewcommand{\section}{\clearpage\section@rig}}{}
%    \end{macrocode}
% \end{macro}
%
% \subsection{Tables and Lists} \label{sec:toc}
%
% These functions are used to decide what page numbers are used for
% the front matter section.
%
% \begin{macro}{\@setletterpagenum}
% \begin{macro}{\@setpostletterpagenum}
% Section 9.9.1, Figure 3, of the CESRM shows an example ``Table of
% Contents'' that begins on page i.  We will take this as the correct
% example.
%    \begin{macrocode}
\newcommand{\@setletterpagenum}{}
\newcommand{\@setpostletterpagenum}{\setcounter{page}{0}}
%    \end{macrocode}
%
% Section 9.8.5 of the CESRM~\cite{ref:cesrm} states that the ``Table of
% Contents'' must start on page ii, contradicting Section 9.9.1 above.  
% However, this does not take into account letters of submittal that are
% longer than one page.  The following code is commented out as it makes 
% no sense to follow this requirement.
%    \begin{macrocode}
%\newcommand{\@setletterpagenum}{}
%\newcommand{\@setpostletterpagenum}{\setcounter{page}{1}}
%    \end{macrocode}
%
% Section 3.1 of the Math guidelines~\cite{ref:mwrg} state that the title 
% page is page i and the ``Table of Contents'' is page ii, because the 
% letter of submittal is an insert, and not part of the report.
%    \begin{macrocode}
\ifthenelse{\boolean{uwwkrpt@math}}{%
  \renewcommand{\@setletterpagenum}{\setcounter{page}{1}}
  \renewcommand{\@setpostletterpagenum}{}
}{}
%    \end{macrocode}
%
% Section 2 of the E\&CE~\cite{ref:ecewrg} and SE~\cite{ref:sewrg} 
% guidelines require that the submittal letter be page ii.  
% The ``Table of Contents'' then follow in logical order, after any
% preliminary sections.
%    \begin{macrocode}
\ifthenelse{\boolean{uwwkrpt@ece} \or \boolean{uwwkrpt@se}}{%
  \renewcommand{\@setletterpagenum}{\setcounter{page}{2}}
  \renewcommand{\@setpostletterpagenum}{}
}{}
%    \end{macrocode}
% \end{macro}
% \end{macro}
%
% \subsubsection{Table of contents}
% In \LaTeX, the default name for a ``Table of Contents'' section is 
% ``Contents''.  This is changed to follow Section 9.9.1 of the 
% CESRM~\cite{ref:cesrm}.
%    \begin{macrocode}
\renewcommand{\contentsname}{Table of Contents}
%    \end{macrocode}
%
% \begin{macro}{\tableofcontents}
% The table should be single-spaced, as |\parskip| should give adequate
% spacing between items.
% The table of contents should list itself in Software Engineering reports.
%    \begin{macrocode}
\newcommand{\toc@intoc}{\relax}
\ifthenelse{\boolean{uwwkrpt@se}}{%
  \renewcommand{\toc@intoc}{%
    \addcontentsline{toc}{section}{Table of Contents}}}{}
\let\tableofcontents@rig\tableofcontents
\renewcommand{\tableofcontents}{%
  \clearpage
  \begin{singlespacing}
  \setlength{\parskip}{0pt}
  \tableofcontents@rig \toc@intoc \par
  \end{singlespacing}
}
%    \end{macrocode}
% \end{macro}
%
% \changes{v2.8}{2012/04/24}{SE Table of Contents lists itself.}
%
% Between the heading of each section, and the right-justified page
% numbers, there should be dotted tab leaders to lead the eye across the
% table.  By default, sections did not have this behaviour, although
% subsections did.  The following code makes it apply to all.
%    \begin{macrocode}
\renewcommand*\l@section[2]{%
    \ifnum \c@tocdepth >\m@ne
      \addpenalty{-\@highpenalty}%
      \vskip 1.0em \@plus\p@
      \setlength\@tempdima{1.5em}%
      \begingroup
        \parindent \z@ \rightskip \@pnumwidth
        \parfillskip -\@pnumwidth
        \leavevmode \bfseries
        \advance\leftskip\@tempdima
        \hskip -\leftskip
        #1\nobreak\
          \leaders\hbox{$\m@th
          \mkern \@dotsep mu\hbox{.}\mkern \@dotsep
          mu$}\hfil\nobreak\hb@xt@\@pnumwidth{\hss #2}\par
        \penalty\@highpenalty
      \endgroup
    \fi%
  }
%    \end{macrocode}
%
% \subsubsection{Lists of stuff}
%
% \begin{macro}{\listoffigures@intoc}
% \begin{macro}{\listoftables@intoc}
% The ``List of Figures'' and ``List of Tables'' should not be 
% mentioned in the Tables of Contents, according to Section 9.9.1 
% of the CESRM~\cite{ref:cesrm}.  This is the default behaviour for
% \LaTeX{}.
%    \begin{macrocode}
\newcommand{\listoffigures@intoc}{\relax}
\newcommand{\listoftables@intoc}{\relax}
%    \end{macrocode}
%
% However, Section 2 of the E\&CE~\cite{ref:ecewrg} and the
% SE~\cite{ref:sewrg} guidelines state that they must be listed.  
% So we add the ``List of Figures'' and ``List of Tables'' to the 
% ``Table of Contents''.
%    \begin{macrocode}
\ifthenelse{\boolean{uwwkrpt@ece} \or \boolean{uwwkrpt@se}}{%
  \renewcommand{\listoffigures@intoc}{%
    \addcontentsline{toc}{section}{List of Figures}}
  \renewcommand{\listoftables@intoc}{%
    \addcontentsline{toc}{section}{List of Tables}}
}{}
%    \end{macrocode}
% \end{macro}
% \end{macro}
%
% \begin{macro}{\listoffigures}
% We ensure that the ``List of Figures'' is on a separate page and
% single-spaced. The spacing provided by |\parskip| is sufficient. 
% Also included is |\listoffigures@intoc| to respect the settings above.
%
% The SE guidlines require that the entries are left-justified and not
% indented. Furthermore, we include the full label, eg ``Figure 1-2''.
%    \begin{macrocode}
\ifthenelse{\boolean{uwwkrpt@se}}{%
  \RequirePackage[titles]{tocloft}
  \renewcommand{\cftsecleader}{\cftdotfill{\cftsubsecdotsep}}
  \setlength{\cftfigindent}{0pt}
  \newlength{\myfiglen}
  \renewcommand{\cftfigpresnum}{\figurename\enspace}
  \renewcommand{\cftfigaftersnum}{:}
  \settowidth{\myfiglen}{\cftfigpresnum\cftfigaftersnum}
  \addtolength{\cftfignumwidth}{\myfiglen}
}{}
\let\listoffigures@rig\listoffigures
\renewcommand{\listoffigures}{%
  \clearpage
  \begin{singlespacing}
  \listoffigures@rig \listoffigures@intoc%
  \end{singlespacing}
}
%    \end{macrocode}
% \end{macro}
%
% \begin{macro}{\listoftables}
% The ``List of Tables'' should behave exactly as the ``List of
% Figures''.
%    \begin{macrocode}
\ifthenelse{\boolean{uwwkrpt@se}}{%
  \setlength{\cfttabindent}{0pt}
  \newlength{\mytablen}
  \renewcommand{\cfttabpresnum}{\tablename\enspace}
  \renewcommand{\cfttabaftersnum}{:}
  \settowidth{\mytablen}{\cfttabpresnum\cfttabaftersnum}
  \addtolength{\cfttabnumwidth}{\mytablen}
}{}
\let\listoftables@rig\listoftables
\renewcommand{\listoftables}{%
  \clearpage
  \begin{singlespacing}
  \listoftables@rig \listoftables@intoc%
  \end{singlespacing}
}
%    \end{macrocode}
% \end{macro}
% \changes{v2.8}{2012/04/24}{Adjust the labels and formatting for the List of
%                            Figures and the List of Tables.}
%
% \subsection{Tables and figures}
%
% Save the original table and figure environments so that they can be
% overridden.  Notice that the |\endtable| command is an implementation
% dependant part of \LaTeX{}.
%    \begin{macrocode}
\let\table@rig\table
\let\endtable@rig\endtable
\let\figure@rig\figure
\let\endfigure@rig\endfigure
%    \end{macrocode}
%
% \begin{environment}{figure}
% \begin{environment}{table}
% According to Section 9.9.3 of the CESRM~\cite{ref:cesrm}, Figures
% and tables must be on their own page after they have been referenced
% in the text.  The only way to guarantee this is to change the default 
% \meta{loc} argument in |\begin{table}|\oarg{loc} to |[p]|.
%    \begin{macrocode}
\renewenvironment{figure}[1][p]{\begin{figure@rig}[#1]}{\end{figure@rig}}
\renewenvironment{table}[1][p]{\begin{table@rig}[#1]}{\end{table@rig}}
%    \end{macrocode}
%
% According to Section 3.4 of the Math guidelines~\cite{ref:mwrg};
% and Section 2 of the E\&CE~\cite{ref:ecewrg} and the
% SE~\cite{ref:sewrg} guidelines, figures and tables must appear
% after they are referenced in the text.  The only way to guarantee this 
% is to change the default \meta{loc} argument to |[htbp]|.  See Section
% \ref{sec:body} for more information.
%
% Software Engineering reports require that table and figure numbering restarts
% in each appendix, and uses a combination of the appendix label and
% table/figure number, eg A-1. To ensure this, we extend this numbering scheme
% to main body sections as well.
%    \begin{macrocode}
\ifthenelse{\boolean{uwwkrpt@ece} 
            \or \boolean{uwwkrpt@math} 
            \or \boolean{uwwkrpt@se}}{%
  \renewenvironment{figure}[1][htbp]{\begin{figure@rig}[#1]}{\end{figure@rig}}
  \renewenvironment{table}[1][htbp]{\begin{table@rig}[#1]}{\end{table@rig}}
}{}
\ifthenelse{\boolean{uwwkrpt@se}}{%
  \RequirePackage{caption}
  \captionsetup{
    font=small,
    labelfont=bf,
    figurewithin=section,
    tablewithin=section
  }
  \renewcommand\thefigure{\thesection-\arabic{figure}}
  \renewcommand\thetable{\thesection-\arabic{table}}
}{}
%    \end{macrocode}
% \end{environment}
% \end{environment}
% \changes{v2.8}{2012/04/24}{Modify the table/figure labels for SE reports.}
% \changes{v2.8.3}{2012/04/24}{Update table/figure labels for SE reports.}
%
% \subsection{References}
%
% Every paper needs a ``References'' section.  Section 9.9.5 of the
% CESRM~\cite{ref:cesrm} does not set any bibliography style.  Section 2
% of the E\&CE~\cite{ref:ecewrg} and the SE~\cite{ref:sewrg} guidelines
% require the use of the IEEE Computer Society
% style~\cite{ref:ieeecsbib}.  The Math guidelines~\cite{ref:mwrg}
% appear to specify a style similar to the IEEE's.
%
% The IEEE Transactions bibliography style is almost identical to that
% of the IEEE Computer Society style.  An implementation of this ships
% with almost every \LaTeX{} installation, so we will call that instead.
% The only difference is that |@ELECTRONIC{}| does not exist as a type
% of citation, but this can be emulated with |@MISC{}|.
%
% For SE reports, the bibliography style should be handled in the .tex file,
% with the ``biblatex'' package.
%    \begin{macrocode}
\ifthenelse{\not \boolean{uwwkrpt@se}}{\bibliographystyle{ieeetr}}{}
%    \end{macrocode}
%
% \changes{v2.8}{2012/04/24}{For SE reports, handle bibliography in the .tex
%                            file.}
%
% In the future, I may consider adding support for the more recent IEEE
% Transactions style, but only after it ships with the major \TeX{}
% distributions.  As well, I would consider using any styles that the
% IEEE Computer Society implement.
%
% \begin{macro}{\bibliography}
% Add the References section to the Table of Contents.  As well, make it 
% single spaced.  
%    \begin{macrocode}
\let\bib@rig\bibliography
\renewcommand{\bibliography}[1]{%
  \clearpage
  \begin{singlespacing}
  \bibliography@intoc \bib@rig{#1}\par
  \end{singlespacing}
}
%    \end{macrocode}
% \end{macro}
%
% \begin{macro}{\bibliography@intoc}
% According to Section 9.9.1 of the CESRM~\cite{ref:cesrm}, the
% References section is actually a section that comes before the
% Appendices.
%    \begin{macrocode}
\newcommand{\refn@me}{References}
\newcommand{\bibliography@intoc}{%
  \renewcommand{\refname}{%
    \addtocounter{section}{1}%
    \arabic{section}\hspace{2.5ex}\refn@me%
    \addcontentsline{toc}{section}{%
      \numberline{\arabic{section}}{\refn@me}}}%
}
%    \end{macrocode}
%
% However the Math~\cite{ref:mwrg}, E\&CE~\cite{ref:ecewrg}, and
% SE~\cite{ref:sewrg} guidelines state that the 
% |\bibliography|\marg{file} should come after |\backmatter|, since 
% it should not have a section number.
%    \begin{macrocode}
\ifthenelse{\boolean{uwwkrpt@ece} 
            \or \boolean{uwwkrpt@math} 
            \or \boolean{uwwkrpt@se}}{%
  \renewcommand{\bibliography@intoc}{%
      \addcontentsline{toc}{section}{\refn@me}}%
}{}
%    \end{macrocode}
% \end{macro}
% \iffalse
%</class>
% \fi
% \iffalse
%<*legacy-ece>
% \fi
%
% \subsection{Legacy code}
%
% The following code is to retain compatibility with the old
% \textsf{uw-ece-workreport} document class.  It merely provides a stub
% that calls \textsf{uw-wkrpt} with the |[ece]| option.
%
% This document class will be depreciated in \textsf{uw-wkrpt} 3.0.
%    \begin{macrocode}
\ClassWarning{uw-ece-workreport}{%
  The `uw-ece-workreport' class is now ^^J%
  deprecated.  Use `\string\usepackage[ece]{uw-wkrpt}' instead}
\DeclareOption*{\PassOptionsToClass {\CurrentOption}{uw-wkrpt}}
\ProcessOptions
\LoadClass[ece]{uw-wkrpt}
\newcommand{\UWECEWorkReportVersion}{2.0}
%    \end{macrocode}
% \iffalse
%</legacy-ece>
% \fi
% \begin{thebibliography}{99}
% \bibitem{ref:lamport}
% L. Lamport and D. Dibby (Illustrator),
% \textit{\LaTeX{}: a document preparation system.}
% Reading, MA: Addison-Wesley, second ed., 1994.
% \bibitem{ref:cesrm} 
% University of Waterloo, Co-operative education \& career services,
% ``Co-operative education student reference manual.''
% \url{http://www.cecs.uwaterloo.ca/manual/}
% (current 24 Apr. 2003.)
% \bibitem{ref:mwrg} 
% University of Waterloo, Math undergrad office,
% ``Faculty of mathematics work report guidelines.''
% \url{http://www.math.uwaterloo.ca/navigation/Current/workreport/index.html}
% (current 26 Apr. 2003.)
% \bibitem{ref:ecewrg} 
% W. M. Loucks PEng, G. H. Freeman, and J.A. Bary PEng,
% ``E\&CE work term report guidelines.''
% \url{http://www.ece.uwaterloo.ca/~wtrc/WrkTrmRpt.html}
% (current 24 Apr. 2003.)
% \bibitem{ref:sewrg} 
% M. Armstrong, J. Atlee, W. M. Loucks PEng, G. H. Freeman, and J.A. Bary PEng,
% ``Software engineering work report guidelines''
% \url{http://www.softeng.uwaterloo.ca/Current/work_report_guidelines.htm}
% (current 24 Apr. 2003.)
% \bibitem{ref:confwtr} 
% W. M. Loucks PEng,
% ``Confidential work term reports.''
% \url{http://www.pads.uwaterloo.ca/Wayne.Loucks/Service/confidential/page1.html}
% (current 26 Apr. 2003.)
% \bibitem{ref:ieeecsbib}
% IEEE Computer Society Press, 
% ``CS Style Guide: References''
% \url{http://www.computer.org/author/style/refer.htm}
% (current 1 Nov. 2001.)
% \end{thebibliography}
% \Finale
\endinput
% vim:et:sw=2 ft=tex
